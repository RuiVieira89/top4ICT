\documentclass[a4paper,12pt]{article}
%

\usepackage{a4wide}
\usepackage{graphicx}
\usepackage{tikz}
\usetikzlibrary{shapes.geometric, arrows}
\usepackage[algo2e]{algorithm2e}
\usepackage{amsmath}
\usepackage{amsfonts}
\usepackage{amssymb}
\usepackage{multicol}
\usepackage{colortbl}
\usepackage{pgf,pgfarrows,pgfnodes,pgfautomata,pgfheaps,pgfshade}
\usepackage{gensymb}
\usepackage[utf8]{inputenc}
\usepackage[T1]{fontenc}
%\usepackage{dsfont}
\usetikzlibrary{shapes,trees}
\usetikzlibrary{graphs}
\usetikzlibrary{positioning}
\usetikzlibrary{arrows}
\usepackage{subcaption}
\usepackage[portuges]{babel}

\begin{document}
%
\title{Trabalho final de AGE e OSD}


\author{R. Vieira\thanks{Departamento de Engenharia Mecânica, Universidade do Minho, {\tt ae5333@alunos.uminho.pt}}}

\maketitle              % typeset the header of the contribution

\begin{abstract}
Este relatório destina-se a apresentar o trabalho das Unidades Curriculares (UCs) de Algoritmos Genéticos e Evolucionários (AGE) e Otimização Sem Derivadas (OSD). Inicia-se com a discrição do problema de otimização, originário da área de trabalho da minha bolsa de investigação. O prblema consiste em optimzar um sistema de teste para placas de circuito impresso quanto ao peso e rigidez. É apresentada uma discussão sobre a estrutura matemática do problema e a abordagem usada na sua resolução. Finaliza-se apresentando as conclusões.
\end{abstract}

\section{Introdução}

As placas de circuito impresso são essenciais numa era de crescente digitilização.

A função objetivo (a minimizar) é:
\begin{equation}\label{eq:fo}
f(x)=(6*qmn*2*b**2)/(np.pi**2*h**2*(s**2+1)**2)*(s**2+ne)
\end{equation}




\end{document}