\documentclass[a4paper,12pt]{article}
%

\usepackage{a4wide}
\usepackage{graphicx}
\usepackage{tikz}
\usetikzlibrary{shapes.geometric, arrows}
\usepackage[algo2e]{algorithm2e}
\usepackage{amsmath}
\usepackage{amsfonts}
\usepackage{amssymb}
\usepackage{multicol}
\usepackage{colortbl}
\usepackage{pgf,pgfarrows,pgfnodes,pgfautomata,pgfheaps,pgfshade}
\usepackage{gensymb}
\usepackage[utf8]{inputenc}
\usepackage[T1]{fontenc}
%\usepackage{dsfont}
\usetikzlibrary{shapes,trees}
\usetikzlibrary{graphs}
\usetikzlibrary{positioning}
\usetikzlibrary{arrows}
\usepackage{subcaption}
\usepackage[portuges]{babel}

\begin{document}
%
\title{Trabalho final de AGE e OSD}


\author{R. Vieira\thanks{Departamento de Engenharia Mecânica, Universidade do Minho, {\tt ae5333@alunos.uminho.pt}}}

\maketitle              % typeset the header of the contribution

\begin{abstract}
Este relatório destina-se a apresentar o trabalho das Unidades Curriculares (UCs) de Algoritmos Genéticos e Evolucionários (AGE) e Otimização Sem Derivadas (OSD). Inicia-se com a discrição do problema de otimização, originário da área de trabalho da minha bolsa de investigação. O prblema consiste em optimzar um sistema de teste para placas de circuito impresso quanto ao peso e rigidez. É apresentada uma discussão sobre a estrutura matemática do problema e a abordagem usada na sua resolução. Finaliza-se apresentando as conclusões.
\end{abstract}

\section{Introdução}

As placas de circuito impresso são essenciais numa era de crescente digitilização.

A função objetivo (a minimizar) é:

\begin{equation}\label{eq:prob}
\begin{split}
\max\;\; &f(x)\\
s.a\;\; & \ell\leq x\leq u
\end{split}
\end{equation}
onde $\ell=(0,0)^T$ e $u=(10,10)^T$.


\begin{equation}\label{eq:fo}
$$x=2\\sqrt{\\frac{2\\pi k T_e}{m_e}} \\left(\\frac{\\Delta E}{k T_e}\\right)^2 a_0^2$$
\end{equation}

%\begin{equation}\label{eq:fo}
%<IPython.core.display.Latex object>
%$$Dconst=\frac{E h^3}{12\left(1-{ne}^2\right)}$$
%\end{equation}
%
%\begin{equation}\label{eq:fo}
%$$k_{}=\frac{k b^4}{Dconst {\pi}^4}$$
%\end{equation}
%
%\begin{equation}\label{eq:fo}
%$$del_T=\frac{T alfa Dconst \left(1+ne\right) {\pi}^2}{b^2}$$
%\end{equation}
%
%\begin{equation}\label{eq:fo}
%$$Wmn=\frac{\frac{b^4}{Dconst {\pi}^4} \left(qmn+del_T \left(m^2 s^2+n^2\right)\right)}{{\left(m^2 s^2+n^2\right)}^2+k_{}}$$
%\end{equation}
%
%\begin{equation}\label{eq:fo}
%$$w0=w0+Wmn sin\left(\frac{m \pi x}{a}\right) sin\left(\frac{n \pi y}{b}\right)$$
%\end{equation}
%
%\begin{equation}\label{eq:fo}
%$$\sigma_{max}=\frac{6qmn\times2 b^2}{{\pi}^2 h^2 {\left(s^2+1\right)}^2} \left(s^2+ne\right)$$
%\end{equation}




\end{document}