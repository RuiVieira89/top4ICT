\documentclass[a4paper,12pt]{article}
%

\usepackage{a4wide}
\usepackage{graphicx}
\usepackage{tikz}
\usetikzlibrary{shapes.geometric, arrows}
\usepackage[algo2e]{algorithm2e}
\usepackage{amsmath}
\usepackage{amsfonts}
\usepackage{amssymb}
\usepackage{multicol}
\usepackage{colortbl}
\usepackage{pgf,pgfarrows,pgfnodes,pgfautomata,pgfheaps,pgfshade}
\usepackage{gensymb}
\usepackage[utf8]{inputenc}
\usepackage[T1]{fontenc}
%\usepackage{dsfont}
\usetikzlibrary{shapes,trees}
\usetikzlibrary{graphs}
\usetikzlibrary{positioning}
\usetikzlibrary{arrows}
\usepackage{subcaption}
\usepackage[portuges]{babel}

\begin{document}
%
\title{Trabalho final de AGE e OSD}


\author{A.~Ismael F.~Vaz\thanks{Algoritmi Research Centre, University of Minho, 4710-057 Braga, Portugal, {\tt aivaz@dps.uminho.pt}}}

\maketitle              % typeset the header of the contribution
%
\begin{abstract}
Este relatório destina-se a apresentar o trabalho das Unidades Curriculares (UCs) de Algoritmos Genéticos e Evolucionários (AGE) e Otimização Sem Derivadas (OSD). Inicia-se com a discrição do problema de otimização, originário da área de trabalho da minha bolsa de investigação. O problema é sobre como maximizar a quantidade de leveduras em iogurtes. É apresentada uma discussão sobre a estrutura matemática do problema e a abordagem usada na sua resolução. Finaliza-se apresentando as conclusões.
\end{abstract}

%
%
%
\section{Introdução}

Os iogurtes são importantes na nossa alimentação e como tal importa maximizar a quantidade de leveduras presentes.

A função objetivo (a maximizar) é:
\begin{equation}\label{eq:fo}
f(x)=\dots
\end{equation}

Pretende-se maximizar a função $f(x)$ descrita na Equação~\eqref{eq:fo} sabendo que as variáveis $x$ (que representam as quantidades de leite e açúcar) são não negativas e não podem exceder a capacidade do recipiente onde serão confecionadas. Como tal temos $0\leq x_1\leq 10$ (quantidade de leite) e $0\leq x_2\leq 10$ (quantidade de açúcar), originando o seguinte problema de otimização:
\begin{equation}\label{eq:prob}
\begin{split}
\max\;\; &f(x)\\
s.a\;\; & \ell\leq x\leq u
\end{split}
\end{equation}
onde $\ell=(0,0)^T$ e $u=(10,10)^T$.

Nas secções seguintes apresenta-se uma breve discrição sobre o problema e qual a melhor abordagem na sua resolução. Concluímos na Secção~\ref{sc:conc}.

\section{Estrutura do problema e técnica mais adequada}

Como as funções envolvidas no problema de otimização~\eqref{eq:prob} são diferenciáveis poderíamos usar um solver que as use ou estime ({\tt fmincon} do MATLAB~\cite{matlab} em vez do {\tt fminunc}, porque temos restrições do tipo limite simples).

No entanto, suspeita-se (como se pode ver na Figura~\ref{fig:land})\footnote{Podem fixar algumas variáveis e desenhar um gráfico a três dimensões com apenas duas das variáveis. Isso permite ter uma ideia do tipo de função. Veja no Apêndice~\ref{A:2} como proceder.} que a função $f(x)$ é multimodal e por isso iremos usar o \emph{solver} mais adequado que seria o {\tt GA} ou o {\tt PSwarm}~\cite{AIFVaz_LNVicente_2009}.

\begin{figure}
\begin{center}
\includegraphics[scale=0.5]{figure.eps}
\end{center}
\caption{\label{fig:land}A função desenhada com $z$ fixo ($z=2$)}
\end{figure}


\section{Solução ótima do problema}

Apesar de não ser (ou ser) o mais adequado, utilizamos os vários \emph{solvers} de AGE e OSD para apresentar uma solução ótima para o problema.

\subsection{AGE -- GA}

Que parâmetros tivemos de alterar/usar... Tamanho da população, etc...

\subsection{OSD -- {\tt fminsearh}}

Podem ter de ignorar algumas restrições e isso pode fazer com que o problema deixe de possuir uma solução finita...

\subsection{OSD -- Pattern Search}


\subsection{OSD -- {\tt PSwarm}}


\section{Conclusões}\label{sc:conc}

Resolvemos ou não o problema e qual a diferença entre os \emph{solvers}. Tive de suar muito para escrever o código que apresento no Apêndice~\ref{A:1}.



\appendix

\section{O código}\label{A:1}

O código que me deu muito trabalho, mas que provavelmente o professor não vai ler.

\section{A figura}\label{A:2}

A figura foi feita com o código MATLAB que se segue e gravada em formato .eps.

\begin{verbatim}
[x,y]=meshgrid(0:0.01:3,0:.01:3);
%plot da função f(x,y,z)=x^2-2*x*z+cos(y)
% fixando o z=2, por exemplo
z=2;
f=@(x,y,z)x^2-2*x*z+cos(y);

for i=1:size(x,1)
    for j=1:size(x,2)
        fval(i,j)=f(x(i,j),y(i,j),z);
    end
end

plot3(x,y,fval);
xlabel('x');
ylabel('y');
title('Função com z=2');
\end{verbatim}


\bibliographystyle{plain}
\bibliography{relatorio}

\end{document}